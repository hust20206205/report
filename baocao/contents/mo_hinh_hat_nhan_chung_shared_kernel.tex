Trong thực tế,      nhiều    bối cảnh giới hạn phụ thuộc lẫn nhau.   Mô hình hợp tác (Partnership)   tạo điều kiện     cho việc giao tiếp và cộng tác giữa các       bối cảnh giới hạn phụ thuộc. Tuy nhiên,  sự phụ thuộc     này dẫn đến mức độ kết hợp cao giữa các nhóm và bối cảnh giới hạn,  dẫn tới mất đi tính độc lập. 

\textit{Lưu ý:    Mô hình hợp tác  không phải là mô hình  của  các mẫu chiến lược trong thiết kế huớng miền.}  


Để giải quyết vấn đề   bối cảnh giới hạn phụ thuộc lẫn nhau chúng ta có mô hình      hạt nhân chung.  Mô hình hạt nhân chung (Shared Kernel) cho phép   các    bối cảnh giới hạn  có phần chia sẻ chung  và  có  ranh giới   phân định rõ ràng.  Từ đó, tách việc quản lí các mô hình hạt nhân chung này một cách độc lập với phần còn lại của bối cảnh giới hạn. Khi cần  thay đổi mà không phải của mô hình hạt nhân chung thì nhóm sẽ   hoạt động độc lập.    Thông thường, mô hình hạt nhân chung được hiện thực hóa bằng các thư viện chung.      Tuy nhiên, chỉ sử dụng mô hình hạt nhân chung nếu quan hệ của các   bối cảnh giới hạn   nhỏ và ổn định   để tránh    quan hệ    phức tạp và ràng buộc  chặt chẽ.

% hình ảnh 
% %! $VD: hình giao như 2 tập hợp - - >

