% %! Các đối tượng và tập hợp giá trị thực thể được gọi là các đối tượng miền, các đối tượng miền này được sử dụng để mô hình hóa dữ liệu trong mô hình miền.
 

% %! Domain Object : https:// thiết kế hướng miền - practitioners.com/domain - object
 

Đối tượng miền

Trong Thiết kế hướng miền (thiết kế hướng miền), một đối tượng miền đề cập đến một đối tượng đại diện cho một khái niệm hoặc một thành phần trong miền vấn đề. Nó là khối xây dựng cơ bản của mô hình miền và được sử dụng để mô hình hóa các thực thể kinh doanh, đối tượng giá trị và dịch vụ tạo nên miền.

Đối tượng miền có thể là một thực thể, là đối tượng miền có nhận dạng duy nhất hoặc đối tượng giá trị, là đối tượng đại diện cho một đặc điểm của miền mà không có nhận dạng riêng.

Các đối tượng miền là trọng tâm của thiết kế hướng miền, vì chúng được sử dụng để triển khai logic nghiệp vụ và trạng thái của miền, đồng thời cung cấp sự thể hiện các khái niệm và đối tượng nghiệp vụ trong miền vấn đề. Chúng được sử dụng để mô hình hóa các quy tắc nghiệp vụ, các ràng buộc và các mối quan hệ xác định hành vi của nghiệp vụ và cung cấp cách gói gọn logic nghiệp vụ phức tạp một cách rõ ràng và ngắn gọn.
 
