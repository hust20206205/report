




Trong mô hình khách hàng - nhà cung cấp, nếu nhà cung cấp có thể thay đổi  linh hoạt không đảm bảo đáp ứng nhu cầu của    khách hàng    thì     khách hàng    cần có giải pháp xử lí.     Mô hình chống đổ vỡ (Anti Corruption Layer)  là một mối quan hệ trong đó bối cảnh giới hạn hạ lưu   sử dụng một lớp để dịch giữa ngôn ngữ của   nó và ngôn ngữ của   bối cảnh giới hạn thượng nguồn.   

Trong mô hình  chống đổ vỡ, mỗi     bối cảnh giới hạn  có mô hình riêng biệt và lớp chống đổ vỡ cần kiến thức về   mô hình    hạ lưu     và thượng nguồn để bảo vệ    hạ lưu  và duy trì tính toàn vẹn. 

%@ Façade 
%@ Adapter
Trong mô hình  chống đổ vỡ      bối cảnh giới hạn hạ lưu  được ký hiệu là  ACL.



% Vẽ lại bản đồ tiếng Việt
% Vẽ lại bản đồ tiếng Việt
% Vẽ lại bản đồ tiếng Việt
% Vẽ lại bản đồ tiếng Việt
% Vẽ lại bản đồ tiếng Việt
% Vẽ lại bản đồ tiếng Việt
% Vẽ lại bản đồ tiếng Việt
% Vẽ lại bản đồ tiếng Việt
% Từ bản đồ lấy vi dụ cho các mô hình
% Từ bản đồ lấy vi dụ cho các mô hình
% Từ bản đồ lấy vi dụ cho các mô hình
% Từ bản đồ lấy vi dụ cho các mô hình
% Từ bản đồ lấy vi dụ cho các mô hình
% Từ bản đồ lấy vi dụ cho các mô hình
% Từ bản đồ lấy vi dụ cho các mô hình
% Từ bản đồ lấy vi dụ cho các mô hình
% Từ bản đồ lấy vi dụ cho các mô hình
% Từ bản đồ lấy vi dụ cho các mô hình
\begin{example} Trong miền vấn đề ngân hàng,     thẻ tín dụng và khoản vay mua nhà không có mối quan hệ. 
    
    \begin{figure}[H]
        
        \centering
        
        \includegraphics[scale = 0.5]{pictures/mo_hinh_rieng_biet_separate_ways/main.drawio.png}
        
        \caption{Ví dụ  mô hình riêng biệt (Separate Ways)  }
        
    \end{figure}
\end{example} 
