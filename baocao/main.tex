% \documentclass{article}

% \documentclass{book}

\documentclass{report} % Chọn cỡ chữ

\usepackage{contents/start/init}

\usepackage{contents/start/vvn}

%%%%%%%%%%%%%%%%%%%%%%%%%%%%%%%%%%

\pagecolor[RGB]{40, 42, 54}% Đặt màu nền

\color[RGB]{18, 161, 24} % Đặt màu chữ

%%%%%%%%%%%%%%%%%%%%%%%%%%%%%%%%%%
\begin{document} % Bắt đầu

%%%%%%%%%%%%%%%%%%%%%%%%%%%%%%%%%%

% \input{contents/trang_bia}

% \input{contents/trang_trang}

% \input{contents/trang_bia}

% \input{contents/trang_trang}

% \input{contents/nhan_xet_cua_giang_vien}

% \includepdf[pages = -]{contents/bao_cao_tien_do_1.pdf}

% \includepdf[pages = -]{contents/bao_cao_tien_do_2.pdf}

%%%%%%%%%%%%%%%%%%%%%%%%%%%%%%%%%%

% \input{contents/muc_luc}

%%%%%%%%%%%%%%%%%%%%%%%%%%%%%%%%%%

% \newpage

\section*{\centering LỜI CẢM ƠN}

\addcontentsline{toc}{section}{LỜI CẢM ƠN}

\blindtext % Tạo văn bản ngẫu nhiên

\newpage



% \input{contents/danh_sach_bang}

% \input{contents/danh_sach_hinh_anh}

% \input{contents/danh_sach_cac_cum_tu_viet_tat}

% \input{contents/danh_sach_cac_thuat_ngu}

%%%%%%%%%%%%%%%%%%%%%%%%%%%%%%%%%%

% \chapter*{\centering MỞ ĐẦU}

% \addcontentsline{toc}{chapter}{MỞ ĐẦU}

% \section*{Lý do chọn đề tài}

% \input{contents/ly_do_chon_de_tai}

% \section*{Đối tượng và phạm vi nghiên cứu}

% \input{contents/doi_tuong_va_pham_vi_nghien_cuu}

% \section*{Tóm tắt nội dung đồ án}

% \input{contents/tom_tat_noi_dung_do_an}

%%%%%%%%%%%%%%%%%%%%%%%%%%%%%%%%%%

% \chapter{Giới thiệu chung}

% \input{contents/gioi_thieu_chung}

% \section{Giới thiệu về bài toán hóa đơn điện tử}

% \input{contents/gioi_thieu_ve_bai_toan_hoa_don_dien_tu}

% \emph{Theo em tìm hiểu có các khái niệm và căn cứ pháp lý liên quan sau đây:}

% \subsection{Hóa đơn}

% \emph{Theo quy định tại khoản 1 Điều 3 Nghị định 123/2020/NĐ - CP:}

% \input{contents/hoa_don}

% \subsection{Hóa đơn điện tử}

% \emph{Theo quy định tại khoản 2 Điều 3 Nghị định 123/2020/NĐ - CP:}

% \input{contents/hoa_don_dien_tu}

% \subsection{Bắt buộc sử dụng hóa đơn điện tử từ 01/07/2022}

% \emph{Theo quy định tại khoản 1 Điều 59 Nghị định 123/2020/NĐ - CP:}

% \input{contents/bat_buoc_su_dung_hoa_don_dien_tu_tu_01072022}

% \subsection{Qui định lưu trữ hóa đơn điện tử}

% \input{contents/luu_tru_hoa_don_dien_tu}

% \subsection{Một số lợi ích của hóa đơn điện tử}

% \input{contents/mot_so_loi_ich_cua_hoa_don_dien_tu}

%%%%%%%%%%%%%%%%%%%%%%%%%%%%%%%%%% @ \subsection{chưa xong}

% Nghiệp vụ?

% UML

%@ \chapter{Yêu cầu nghiệp vụ}

%@ \input{contents/yeu_cau_nghiep_vu}

%@ \subsection {Yêu cầu nghiệp vụ của bài toán phụ}

%@ \input{contents/yeu_cau_nghiep_vu_cua_bai_toan_phu}

%@ \subsubsection{Các chức năng tổng quan của bài toán phụ}

%@ \input{contents/cac_chuc_nang_tong_quan_cua_bai_toan_phu}

%@ \subsection{Yêu cầu nghiệp vụ chưa xong}

%@ \input{contents/yeu_cau_nghiep_vu_chua_xong}

%%%%%%%%%%%%%%%%%%%%%%%%%%%%%%%%%% @ \subsection{chưa xong}

%%%%%%%%%%%%%%%%%%%%%%%%%%%%%%%%%% @ \subsection{chưa xong}

%! chưa chắc vì mình có mẫu kt sẽ khác

%! chưa chắc vì mình có mẫu kt sẽ khác

%@ \chapter{Phân tích thiết kế hệ thống}

%@ \section{UML Use Case Diagrams}

%@ \section{UML Activity Diagrams}

%@ \section{UML Sequence Diagrams}

%@ \section{UML Class Diagrams}

%%%%%%%%%%%%%%%%%%%%%%%%%%%%%%%%%% @ \subsection{chưa xong}

%%%%%%%%%%%%%%%%%%%%%%%%%%%%%%%%%% @ \subsection{chưa xong}

%%%%%%%%%%%%%%%%%%%%%%%%%%%%%%%%%% @ \subsection{chưa xong}

%%%%%%%%%%%%%%%%%%%%%%%%%%%%%%%%%% @ \subsection{chưa xong}

%%%%%%%%%%%%%%%%%%%%%%%%%%%%%%%%%% @ \subsection{chưa xong}

%%%%%%%%%%%%%%%%%%%%%%%%%%%%%%%%%% @ \subsection{chưa xong}

%%%%%%%%%%%%%%%%%%%%%%%%%%%%%%%%%% @ \subsection{chưa xong}

%%%%%%%%%%%%%%%%%%%%%%%%%%%%%%%%%% @ \subsection{chưa xong}

%%%%%%%%%%%%%%%%%%%%%%%%%%%%%%%%%% @ \subsection{chưa xong}

%%%%%%%%%%%%%%%%%%%%%%%%%%%%%%%%%% @ \subsection{chưa xong}

%%%%%%%%%%%%%%%%%%%%%%%%%%%%%%%%%%

% \newpage

% \newpage

% \newpage

% \newpage

% \newpage

% \newpage

% \newpage

% \newpage

% \newpage

% \newpage

% \newpage

% \newpage

% \newpage

% \newpage

% \section{Giới thiệu về kiến trúc vi dịch vụ}

% \subsection{Kiến trúc nguyên khối}

% \input{contents/kien_truc_nguyen_khoi}

% \subsection{Kiến trúc vi dịch vụ}

% \input{contents/kien_truc_vi_dich_vu}

% \subsection{Một số đặc điểm và ưu điểm của kiến trúc vi dịch vụ}

% \input{contents/mot_so_dac_diem_va_uu_diem_cua_kien_truc_vi_dich_vu}

% \newpage

% \newpage

% \newpage

% \newpage

% \newpage

% \newpage

% \newpage

% \newpage

% \newpage

% \newpage

% \newpage

% \newpage

% \newpage

% \subsection{Một số nhược điểm và thách thức của kiến trúc vi dịch vụ}

% \input{contents/mot_so_nhuoc_diem_va_thach_thuc_cua_kien_truc_vi_dich_vu}

%%%%%%%%%%%%%%%%%%%%%%%%%%%%%%%%%%

% \section{Giới thiệu về thiết kế hướng miền}

% 





$\Rightarrow$ Kiến trúc vi dịch vụ giải quyết những thách thức và hỗ trợ doanh nghiệp chuyển đổi dễ dàng. Tuy nhiên, để xây dựng được kiến trúc vi dịch vụ tốt, cần phải tạo ra các dịch vụ nhỏ phù hợp và duy trì tính độc lập. Trong đồ án này, em sử dụng thiết kế hướng miền để phân tích và xây dựng kiến trúc vi dịch vụ. Thiết kế hướng miền xác định và tổ chức các dịch vụ dựa trên việc hiểu rõ về lĩnh vực kinh doanh, giúp dự án phản ánh đúng các quy trình và quy tắc kinh doanh.

% \subsection{Đôi nét về thiết kế hướng miền (Domain Driven Design)}

% \input{contents/doi_net_ve_thiet_ke_huong_mien}

% \input{contents/chuyen_gia_nganh}

% \subsection{Định nghĩa miền (Domain)}

% Hệ thống được tạo ra để xử lý sự phức tạp trong cuộc sống hiện đại. Việc phát triển hệ thống liên kết chặt chẽ với một số khía cạnh cụ thể trong cuộc sống của chúng ta.

Trong domain driven design, \emph{miền (Domain)} đề cập đến phạm vi kiến thức và vấn đề cụ thể mà hệ thống xử lý.

Xét theo góc độ kinh doanh và góc độ hệ thống:

\begin{itemize}

\item Về góc độ kinh doanh: miền đại diện cho một lĩnh vực hoặc ngành mà doanh nghiệp hoạt động.

\item Về góc độ hệ thống: miền có thể coi là đại diện cho không gian vấn đề của hệ thống.

\end{itemize}

Hệ thống cần phản ánh đúng miền và hiện thực hóa chính xác miền.

\begin{example} \emph{Trong đồ án này, miền được xác định là bài toán giải pháp hóa đơn điện tử.}

\end{example}

% \subsection{Giới thiệu về các mẫu chiến lược và các mẫu kỹ thuật}

% \input{contents/chi_tiet_va_ap_dung_thiet_ke_huong_mien}

%%%%%%%%%%%%%%%%%%%%%%%%%%%%%%%%%%
%%%%%%%%%%%%%%%%%%%%%%%%%%%%%%%%%%
%%%%%%%%%%%%%%%%%%%%%%%%%%%%%%%%%%
%%%%%%%%%%%%%%%%%%%%%%%%%%%%%%%%%%
%%%%%%%%%%%%%%%%%%%%%%%%%%%%%%%%%%
%%%%%%%%%%%%%%%%%%%%%%%%%%%%%%%%%%
%%%%%%%%%%%%%%%%%%%%%%%%%%%%%%%%%%
%%%%%%%%%%%%%%%%%%%%%%%%%%%%%%%%%%
%%%%%%%%%%%%%%%%%%%%%%%%%%%%%%%%%%
%%%%%%%%%%%%%%%%%%%%%%%%%%%%%%%%%%
%%%%%%%%%%%%%%%%%%%%%%%%%%%%%%%%%%
%%%%%%%%%%%%%%%%%%%%%%%%%%%%%%%%%%
%%%%%%%%%%%%%%%%%%%%%%%%%%%%%%%%%%
%%%%%%%%%%%%%%%%%%%%%%%%%%%%%%%%%%
%%%%%%%%%%%%%%%%%%%%%%%%%%%%%%%%%%
%%%%%%%%%%%%%%%%%%%%%%%%%%%%%%%%%%
%%%%%%%%%%%%%%%%%%%%%%%%%%%%%%%%%%
%%%%%%%%%%%%%%%%%%%%%%%%%%%%%%%%%%
%%%%%%%%%%%%%%%%%%%%%%%%%%%%%%%%%%

\chapter{Các mẫu chiến lược}

% Các mẫu chiến lược phân tích nghiệp vụ kinh doanh sau đó đưa ra việc phân chia các thành phần và hiểu mối quan hệ của các thành phần đó. Từ đó, các mẫu chiến lược giúp xác định các thành phần quan trọng của hệ thống, đảm bảo kiến trúc phần mềm phản ánh đúng các yêu cầu kinh doanh. Từ việc phân chia hệ thống thành các thành phần nhỏ, chúng ta có thể tạo ra hệ thống mở rộng dễ dàng, phát triển linh hoạt theo nhu cầu kinh doanh.

Các mẫu chiến lược bao gồm:

\begin{itemize}

% các mục nhỏ ben dưới

% các mục nhỏ ben dưới

% các mục nhỏ ben dưới

% các mục nhỏ ben dưới

% các mục nhỏ ben dưới

% các mục nhỏ ben dưới

\item Muc1

\item Muc2

\item Muc1

\item Muc2

\item Muc1

\item Muc2

\item Muc1

\item Muc2

\end{itemize}

% nội dung trang lớn lên để hết giấy

% nội dung trang lớn lên để hết giấy

% nội dung trang lớn lên để hết giấy

% nội dung trang lớn lên để hết giấy

% nội dung trang lớn lên để hết giấy

\begin{figure}[H]

\centering

\includegraphics[scale = 0.9]{pictures/cac_mau_chien_luoc/temp.png}

\caption{Sơ đồ về các thành phần trong mô hình chiến lược}

\end{figure}

%!<! - - $ Vẽ lại sau: - - >

%!<! - - $ Vẽ lại sau: - - >

%!<! - - $ Vẽ lại sau: - - >

%!<! - - $ Vẽ lại sau: - - >

%!<! - - $ Vẽ lại sau: - - >

%!<! - - $ Vẽ lại sau: - - >

%!<! - - $ Vẽ lại sau: - - >

%!<! - - $ Vẽ lại sau: - - >

%!<! - - $ Vẽ lại sau: - - >

%!<! - - $ Vẽ lại sau: - - >



% \newpage

% \newpage

% \newpage

% \newpage

% \newpage

% \newpage

% \newpage

% \newpage

% \newpage

% \newpage

% \newpage

% \newpage

% \newpage

% \section{Miền phụ (Sub - Domain)}

% Một miền lớn được tạo thành từ nhiều \emph{miền phụ (Sub - Domain)}. Trong thực tế, một miền kinh doanh phức tạp không thể có một chuyên gia ngành có kiến thức về tất cả các miền phụ.

\begin{example} Trong miền thương mại điện tử lớn có thể có một số miền phụ sau:

\begin{itemize}

\item \textbf{Miền phụ quản lý hàng tồn kho:} liên quan đến việc quản lý sản phẩm trong kho hàng.

\item \textbf{Miền phụ quản lý khách hàng:} liên quan đến việc quản lý tài khoản khách hàng.

\item \textbf{Miền phụ vận chuyển:} liên quan đến việc quản lý việc vận chuyển giao hàng.

\end{itemize}

\end{example}


% \subsection{Phân loại các miền phụ}

% \input{contents/phan_loai_cac_mien_phu}

% \subsubsection{Miền phụ chung (Generic Subdomain)}

% Miền phụ chung cung cấp các giải pháp có sẵn mà doanh nghiệp có thể mua. Miền phụ chung có thể được tìm thấy trên nhiều ngành. Doanh nghiệp không thể đạt được bất kỳ lợi thế cạnh tranh nào so với đối thủ bằng cách thực hiện những điều khác biệt trong miền phụ chung.

\begin{example} Các miền phụ chung \textit{"quản lý nhân sự"} hay \textit{"quản lý cơ sở vật chất"} không tạo thêm bất kỳ giá trị khác biệt nào cho doanh nghiệp.

\end{example}



% \subsubsection{Miền phụ cốt lõi (Core Subdomain)}

% Miền phụ cốt lõi là phần quan trọng và có giá trị nhất của hệ thống. Miền phụ cốt lõi giúp phân biệt các doanh nghiệp và làm cho các doanh nghiệp có giá trị. Miền phụ cốt lõi tập trung vào mục tiêu và yêu cầu của khách hàng với doanh nghiệp, từ đó quyết định sự thành công của doanh nghiệp. Vì vậy, mỗi doanh nghiệp luôn tìm cách thực hiện những điều khác biệt trong các miền phụ cốt lõi này để đạt được lợi thế so với đối thủ cạnh tranh.

\begin{example} Trong miền thẻ tín dụng, miền phụ cốt lõi có thể là \textit{"phát hành thẻ"} chịu trách nhiệm về quá trình phát hành thẻ tín dụng cho khách hàng. Miền phụ cốt lõi này bao gồm các nhiệm vụ như: thu thập thông tin khách hàng, thực hiện kiểm tra tín dụng, kích hoạt thẻ, \dots

\end{example}

% \subsubsection{Miền phụ hỗ trợ (Supporting Subdomain)}

% \input{contents/mien_phu_ho_tro_supporting_subdomain}

% \newpage

% \newpage

% \newpage

% \newpage

% \newpage

% \newpage

% \newpage

% \newpage

% \newpage

% \newpage

% \newpage

% \newpage

% \newpage

% \newpage

% \newpage

% \newpage

% \newpage

% \newpage

% \subsection{Cách xác định các miền phụ}

% \input{contents/cach_xac_dinh_cac_mien_phu}

% \newpage

% \newpage

% \newpage

% \newpage

% \newpage

% \newpage

% \newpage

% \newpage

% \newpage

% \newpage

% \newpage

% \newpage

% \newpage

% \newpage

% \subsection{Áp dụng phân loại miền phụ trong đồ án này}

% % %!<! - - Hướng dẫn: 5/3 - - >

% %!<! - - Hướng dẫn: 5/3 - - >

% %!<! - - Hướng dẫn: 5/3 - - >

% %!<! - - Hướng dẫn: 5/3 - - >

% %!<! - - Hướng dẫn: 5/3 - - >

% %!<! - - Hướng dẫn: 5/3 - - >

% %!<! - - Hướng dẫn: 5/3 - - >

% %!<! - - Hướng dẫn: 5/3 - - >

% %!<! - - Hướng dẫn: 5/3 - - >

% %!<! - - Hướng dẫn: 5/3 - - >

% %!<! - - Hướng dẫn: 5/3 - - >

% %!<! - - Hướng dẫn: 5/3 - - >

% %!<! - - Hướng dẫn: 5/3 - - >

% %!<! - - Hướng dẫn: 5/3 - - >

% %!<! - - Hướng dẫn: 5/3 - - >

% %!<! - - Hướng dẫn: 5/3 - - >

% ChatGPT?

% Ứng dụng thiết kế hướng miền với hóa đơn điện tử thì miền phụ hỗ trợ có thể là gì?

\subsubsection{Áp dụng phân loại miền phụ trong đồ án này}

\subsubsection{Áp dụng phân loại miền phụ trong đồ án này}

\subsubsection{Áp dụng phân loại miền phụ trong đồ án này}

\subsubsection{Áp dụng phân loại miền phụ trong đồ án này}

\subsubsection{Áp dụng phân loại miền phụ trong đồ án này}

\subsubsection{Áp dụng phân loại miền phụ trong đồ án này}

\subsubsection{Áp dụng phân loại miền phụ trong đồ án này}

\subsubsection{Áp dụng phân loại miền phụ trong đồ án này}

\subsubsection{Áp dụng phân loại miền phụ trong đồ án này}

\subsubsection{Áp dụng phân loại miền phụ trong đồ án này}

\subsubsection{Áp dụng phân loại miền phụ trong đồ án này}

\subsubsection{Áp dụng phân loại miền phụ trong đồ án này}

\subsubsection{Áp dụng phân loại miền phụ trong đồ án này}



% \section{Mô hình miền (Domain Models)}

% Để tạo một phần mềm tốt, chúng ta cần phải hiểu rõ về phần mềm đó. Trong thiết kế hướng miền để có thể hiểu miền nhanh, chúng ta cần tạo ra các mô hình miền. Mô hình miền (Domain Models) là kiến thức có tổ chức và có cấu trúc về miền phù hợp để giải quyết vấn đề kinh doanh. Mục tiêu của mô hình miền là cung cấp rõ ràng, ngắn gọn và chính xác về miền làm cơ sở để hệ thống giải quyết vấn đề kinh doanh.

% \begin{example} Trong đồ án này, mô hình miền của em bao gồm các yêu cầu nghiệp vụ và các sơ đồ: UML Use Case Diagrams, UML Class Diagrams,\dots kĩ thuật ở phần sau

% \end{example}

% \section{Bối cảnh giới hạn (Bounded Context)}

% Một miền cần chia đủ nhỏ để phù hợp với một nhóm cụ thể. Để đạt được điều này, chúng ta cần xác định rõ ranh giới giữa các ngữ cảnh. \emph{Bối cảnh giới hạn (Bounded Context)} giúp xác định rõ các ranh giới, chia miền thành các phần độc lập để giải quyết sự phức tạp trong mô hình doanh nghiệp. Bối cảnh giới hạn tạo ra các mô hình khác nhau cho các lĩnh vực khác nhau của miền. Bối cảnh giới hạn thể hiện phạm vi kinh doanh của dịch vụ.

\begin{figure}[H]

\centering

\includegraphics[scale = 1]{pictures/boi_canh_gioi_han/main.png}

\caption{Ví dụ về bối cảnh giới hạn trong một ngân hàng}

\end{figure}

\subsubsection{Cách xác định bối cảnh giới hạn}

Để có thể xác định được bối cảnh giới hạn chúng ta có thể xem xét:

\begin{itemize}

\item Dựa vào việc phân chia các miền phụ.

\item Dựa vào sơ đồ cấu trúc tổ chức các phòng ban của doanh nghiệp.

\item Dựa vào modules của các ứng dụng kiến trúc nguyên khối (nếu việc phân chia tốt).

\item Dựa vào trách nhiệm và hoạt động của chuyên gia ngành.

\end{itemize}

\subsubsection{Áp dụng xác định bối cảnh giới hạn trong đồ án này}

\subsubsection{Áp dụng xác định bối cảnh giới hạn trong đồ án này}

\subsubsection{Áp dụng xác định bối cảnh giới hạn trong đồ án này}

\subsubsection{Áp dụng xác định bối cảnh giới hạn trong đồ án này}

\subsubsection{Áp dụng xác định bối cảnh giới hạn trong đồ án này}

\subsubsection{Áp dụng xác định bối cảnh giới hạn trong đồ án này}

\subsubsection{Áp dụng xác định bối cảnh giới hạn trong đồ án này}

\subsubsection{Áp dụng xác định bối cảnh giới hạn trong đồ án này}

\subsubsection{Áp dụng xác định bối cảnh giới hạn trong đồ án này}

\subsubsection{Áp dụng xác định bối cảnh giới hạn trong đồ án này}

\subsubsection{Áp dụng xác định bối cảnh giới hạn trong đồ án này}

\subsubsection{Áp dụng xác định bối cảnh giới hạn trong đồ án này}

%!<! - - Hướng dẫn 5/10 - - >

%!<! - - Hướng dẫn 5/10 - - >

%!<! - - Hướng dẫn 5/10 - - >

%!<! - - Hướng dẫn 5/10 - - >

%!<! - - Hướng dẫn 5/10 - - >

%!<! - - Hướng dẫn 5/10 - - >

%!<! - - Hướng dẫn 5/10 - - >

%!<! - - Hướng dẫn 5/10 - - >

%!<! - - Hướng dẫn 5/10 - - >

% \section{Ngôn ngữ chung (Ubiquitous Language)}

% \input{contents/ngon_ngu_chung}
%@ Tích hợp liên tục
%@ Tích hợp liên tục
%@ Tích hợp liên tục
%@ Tích hợp liên tục
%@ Tích hợp liên tục
% \section{Bản đồ bối cảnh (Context Maps)}
% Các bối cảnh bị giới hạn phải độc lập trong bối cảnh riêng và có mô hình miền riêng, nhưng các bối cảnh bị giới hạn cần tương tác, giao tiếp để trao đổi thông tin. Vì vậy các bối cảnh bị giới hạn có thể có mối quan hệ với nhau. Những mối quan hệ này cần được quản lý chặt chẽ để hoạt động độc lập, nhất quán và linh hoạt. Do đó, cần phải ghi lại các mối quan hệ thông qua việc sử dụng bản đồ bối cảnh. \emph{Bản đồ bối cảnh (Context Maps)} là sự thể hiện trực quan của hệ thống, thể hiện các thành phần và mối quan hệ giữa các thành phần.

\begin{figure}[H]

\centering

\includegraphics[scale = 0.4]{pictures/ban_do_boi_canh/main.drawio.png}

\caption{Ví dụ bản đồ bối cảnh trong 1 ngân hàng}

\end{figure}

%! Vẽ lại tiếng Việt

%! Vẽ lại tiếng Việt

%! Vẽ lại tiếng Việt

%! Vẽ lại tiếng Việt

%! Vẽ lại tiếng Việt

%! Vẽ lại tiếng Việt

%! Vẽ lại tiếng Việt

%! Vẽ lại tiếng Việt

%! Vẽ lại tiếng Việt

%! Vẽ lại tiếng Việt
\section{Các mối quan hệ bối cảnh giới hạn}
% Có 3 loại mối quan hệ bối cảnh giới hạn là:

\begin{itemize}
    \item Mối quan hệ đối xứng (Symmetric Relationship)
          \textbf{Mô tả:} 

    \item Mối quan hệ bất đối xứng (Asymmetric Relationship)
          \textbf{Mô tả:} 

    \item Mối quan hệ 1 - nhiều (One to Many Relationship)
          \textbf{Mô tả:} 

\end{itemize}

% 

% \subsection{Mối quan hệ đối xứng (Symmetric Relationship)}
% \subsubsection{Mô hình riêng biệt (Separate Ways)}
%  Mô hình riêng biệt (Separate Ways) khi các bối cảnh giới hạn     có quan hệ riêng biệt, không có sự phụ thuộc. Vì vậy,  các   bối cảnh giới hạn   này     có  ngôn ngữ,  mô hình,  mục đích độc lập  và        thực thi riêng biệt.  Các nhóm phát triển không phải cộng tác hay phối hợp  với nhau từ đó   đem lại lợi ích  dễ dàng bảo trì và mở rộng hệ thống.

\begin{example} Trong miền vấn đề ngân hàng,     thẻ tín dụng và khoản vay mua nhà không có mối quan hệ. 
    
\begin{figure}[H]

    \centering
    
    \includegraphics[scale = 0.5]{pictures/mo_hinh_rieng_biet_separate_ways/main.drawio.png}
    
    \caption{Ví dụ  mô hình riêng biệt (Separate Ways)  }
    
    \end{figure}
\end{example} 

% \subsubsection{Mô hình hạt nhân chung (Shared Kernel)}
% Trong thực tế,      nhiều    bối cảnh giới hạn phụ thuộc lẫn nhau.   Mô hình hợp tác (Partnership)   tạo điều kiện     cho việc giao tiếp và cộng tác giữa các       bối cảnh giới hạn phụ thuộc. Tuy nhiên,  sự phụ thuộc     này dẫn đến mức độ kết hợp cao giữa các nhóm và bối cảnh giới hạn,  dẫn tới mất đi tính độc lập. 

\textit{Lưu ý:    Mô hình hợp tác  không phải là mô hình  của  các mẫu chiến lược trong thiết kế huớng miền.}  


Để giải quyết vấn đề   bối cảnh giới hạn phụ thuộc lẫn nhau chúng ta có mô hình      hạt nhân chung.  Mô hình hạt nhân chung (Shared Kernel) cho phép   các    bối cảnh giới hạn  có phần chia sẻ chung  và  có  ranh giới   phân định rõ ràng.  Từ đó, tách việc quản lí các mô hình hạt nhân chung này một cách độc lập với phần còn lại của bối cảnh giới hạn. Khi cần  thay đổi mà không phải của mô hình hạt nhân chung thì nhóm sẽ   hoạt động độc lập.    Thông thường, mô hình hạt nhân chung được hiện thực hóa bằng các thư viện chung.      Tuy nhiên, chỉ sử dụng mô hình hạt nhân chung nếu quan hệ của các   bối cảnh giới hạn   nhỏ và ổn định   để tránh    quan hệ    phức tạp và ràng buộc  chặt chẽ.

% hình ảnh 
% %! $VD: hình giao như 2 tập hợp - - >



\subsection{Mối quan hệ bất đối xứng (Asymmetric Relationship)}
Trong mối quan hệ bất đối xứng, một bối cảnh giới hạn có sự phụ thuộc vào một bối cảnh giới hạn khác. Mối quan hệ này được mô tả bằng cách gán vai trò cho bối cảnh giới hạn:

\begin{itemize}

    \item \textbf{Bối cảnh giới hạn thượng nguồn (Upstream):}
          \begin{itemize}


              \item     Bối cảnh giới hạn cung cấp cho bối cảnh giới hạn khác.
              \item Ký hiệu:      U
          \end{itemize}

    \item \textbf{Bối cảnh giới hạn hạ lưu (Downstream):}
          \begin{itemize}
              \item    Bối cảnh giới hạn phụ thuộc vào bối cảnh giới hạn khác.
              \item Ký hiệu:     D
          \end{itemize}


\end{itemize}

\begin{example} Mối quan hệ bất đối xứng giữa      bối cảnh giới hạn A và     bối cảnh giới hạn B.
\begin{itemize}
    \item      Bối cảnh giới hạn  A ràng buộc với      bối cảnh giới hạn  B

    \item      Bối cảnh giới hạn  A đóng vai trò là bối cảnh giới hạn hạ lưu (Downstream)

    \item      Bối cảnh giới hạn  B đóng vai trò là bối cảnh giới hạn thượng nguồn (Upstream)

    \item      Bối cảnh giới hạn    A có kiến thức về các mô hình trong bối cảnh giới hạn B

    \item      Bối cảnh giới hạn  B không có bất kỳ kiến thức nào về mô hình trong bối cảnh giới hạn A
\end{itemize}



\begin{figure}[H]

    \centering

    \includegraphics[scale = 0.5]{pictures/moi_quan_he_bat_doi_xung/main.drawio.png}

    \caption{Ví dụ  mối quan hệ bất đối xứng }

\end{figure}


\end{example}
\subsubsection{Mô hình khách hàng - nhà cung cấp (Customer - Supplier)}
%  Mô hình khách hàng - nhà cung cấp (Customer - Supplier)     được thể hiện       rằng  bối cảnh giới hạn thượng nguồn đáp ứng nhu cầu của bối cảnh giới hạn hạ lưu.


Khi đó:


\begin{itemize}
    \item       Bối cảnh giới hạn   thượng nguồn  được gọi là  nhà cung cấp.
    \item       Bối cảnh giới hạn   hạ lưu  được gọi là khách hàng.
\end{itemize}





Trong thực tế,   nhóm   phát triển   nhà cung cấp luôn tham khảo ý kiến của  nhóm   phát triển    khách hàng và    có bộ     kiểm thử     để đảm bảo rằng dịch vụ của      nhà cung cấp đáp ứng được yêu cầu của      khách hàng.    
\subsubsection{Mô hình tuân thủ (Conformist)}
% Trong mô hình khách hàng - nhà cung cấp, nếu nhà cung cấp thực hiện tốt yêu cầu thì khách hàng cần tuân thủ chặt chẽ. Mô hình tuân thủ (Conformist) là một mối quan hệ trong đó bối cảnh bị giới hạn hạ lưu áp dụng mô hình, ngôn ngữ chung và các khái niệm của bối cảnh bị giới hạn thượng nguồn.

Trong mô hình tuân thủ bối cảnh bị giới hạn hạ lưu được ký hiệu là CF.

%! $VD: - - >

%! $VD: A - CF - U - B - - >

%! $VD: A - users(id, name) - B cũng users(id, name) - - >

% Vẽ lại bản đồ tiếng Việt

% Vẽ lại bản đồ tiếng Việt

% Vẽ lại bản đồ tiếng Việt

% Vẽ lại bản đồ tiếng Việt

% Vẽ lại bản đồ tiếng Việt

% Vẽ lại bản đồ tiếng Việt

% Vẽ lại bản đồ tiếng Việt

% Vẽ lại bản đồ tiếng Việt

% Từ bản đồ lấy vi dụ cho các mô hình

% Từ bản đồ lấy vi dụ cho các mô hình

% Từ bản đồ lấy vi dụ cho các mô hình

% Từ bản đồ lấy vi dụ cho các mô hình

% Từ bản đồ lấy vi dụ cho các mô hình

% Từ bản đồ lấy vi dụ cho các mô hình

% Từ bản đồ lấy vi dụ cho các mô hình

% Từ bản đồ lấy vi dụ cho các mô hình

% Từ bản đồ lấy vi dụ cho các mô hình

% Từ bản đồ lấy vi dụ cho các mô hình

\begin{example} Trong miền vấn đề ngân hàng, thẻ tín dụng và khoản vay mua nhà không có mối quan hệ.

\begin{figure}[H]

\centering

\includegraphics[scale = 0.5]{pictures/mo_hinh_rieng_biet_separate_ways/main.drawio.png}

\caption{Ví dụ mô hình riêng biệt (Separate Ways)}

\end{figure}

\end{example}


\subsubsection{Mô hình chống đổ vỡ (Anti Corruption Layer)}
% 




Trong mô hình khách hàng - nhà cung cấp, nếu nhà cung cấp có thể thay đổi  linh hoạt không đảm bảo đáp ứng nhu cầu của    khách hàng    thì     khách hàng    cần có giải pháp xử lí.     Mô hình chống đổ vỡ (Anti Corruption Layer)  là một mối quan hệ trong đó bối cảnh giới hạn hạ lưu   sử dụng một lớp để dịch giữa ngôn ngữ của   nó và ngôn ngữ của   bối cảnh giới hạn thượng nguồn.   

Trong mô hình  chống đổ vỡ, mỗi     bối cảnh giới hạn  có mô hình riêng biệt và lớp chống đổ vỡ cần kiến thức về   mô hình    hạ lưu     và thượng nguồn để bảo vệ    hạ lưu  và duy trì tính toàn vẹn. 

%@ Façade 
%@ Adapter
Trong mô hình  chống đổ vỡ      bối cảnh giới hạn hạ lưu  được ký hiệu là  ACL.



% Vẽ lại bản đồ tiếng Việt
% Vẽ lại bản đồ tiếng Việt
% Vẽ lại bản đồ tiếng Việt
% Vẽ lại bản đồ tiếng Việt
% Vẽ lại bản đồ tiếng Việt
% Vẽ lại bản đồ tiếng Việt
% Vẽ lại bản đồ tiếng Việt
% Vẽ lại bản đồ tiếng Việt
% Từ bản đồ lấy vi dụ cho các mô hình
% Từ bản đồ lấy vi dụ cho các mô hình
% Từ bản đồ lấy vi dụ cho các mô hình
% Từ bản đồ lấy vi dụ cho các mô hình
% Từ bản đồ lấy vi dụ cho các mô hình
% Từ bản đồ lấy vi dụ cho các mô hình
% Từ bản đồ lấy vi dụ cho các mô hình
% Từ bản đồ lấy vi dụ cho các mô hình
% Từ bản đồ lấy vi dụ cho các mô hình
% Từ bản đồ lấy vi dụ cho các mô hình
\begin{example} Trong miền vấn đề ngân hàng,     thẻ tín dụng và khoản vay mua nhà không có mối quan hệ. 
    
    \begin{figure}[H]
        
        \centering
        
        \includegraphics[scale = 0.5]{pictures/mo_hinh_rieng_biet_separate_ways/main.drawio.png}
        
        \caption{Ví dụ  mô hình riêng biệt (Separate Ways)  }
        
    \end{figure}
\end{example} 


\subsection{Mối quan hệ 1 - nhiều (One to Many Relationship)}
% \input{contents/moi_quan_he_1_nhieu_one_to_many_relationship}
\subsubsection{Dịch vụ máy chủ mở (Open Host Service)}
% 
Dịch vụ máy chủ mở (Open Host Service) là nhà cung cấp trong mô hình khách hàng - nhà cung cấp, dịch vụ máy chủ mở hiển thị một API công khai cho các bối cảnh bị giới hạn khác sử dụng chức năng của nhà cung cấp.

Trong bản đồ bối cảnh, dịch vụ máy chủ mở được ký hiệu là OHS.

% Vẽ lại bản đồ tiếng Việt

% Vẽ lại bản đồ tiếng Việt

% Vẽ lại bản đồ tiếng Việt

% Vẽ lại bản đồ tiếng Việt

% Vẽ lại bản đồ tiếng Việt

% Vẽ lại bản đồ tiếng Việt

% Vẽ lại bản đồ tiếng Việt

% Vẽ lại bản đồ tiếng Việt

% Từ bản đồ lấy vi dụ cho các mô hình

% Từ bản đồ lấy vi dụ cho các mô hình

% Từ bản đồ lấy vi dụ cho các mô hình

% Từ bản đồ lấy vi dụ cho các mô hình

% Từ bản đồ lấy vi dụ cho các mô hình

% Từ bản đồ lấy vi dụ cho các mô hình

% Từ bản đồ lấy vi dụ cho các mô hình

% Từ bản đồ lấy vi dụ cho các mô hình

% Từ bản đồ lấy vi dụ cho các mô hình

% Từ bản đồ lấy vi dụ cho các mô hình

\begin{example} Trong miền vấn đề ngân hàng, thẻ tín dụng và khoản vay mua nhà không có mối quan hệ.

\begin{figure}[H]

\centering

\includegraphics[scale = 0.5]{pictures/mo_hinh_rieng_biet_separate_ways/main.drawio.png}

\caption{Ví dụ mô hình riêng biệt (Separate Ways)}

\end{figure}

\end{example}


\subsubsection{Ngôn ngữ được xuất bản (Published Language)}
%  
Khi   ngôn ngữ chung ở     dịch vụ máy chủ mở  được các nhóm phát triển trong     bối cảnh giới hạn  hạ lưu chấp nhận.   Ngôn ngữ chung này được gọi là  ngôn ngữ được xuất bản (Published Language). Ngôn ngữ  được xuất bản có lợi ích là tính thống nhất trong hệ thống tuy nhiên cần phân tích kĩ  vì nó có thể tạo ra  sự nhầm lẫn trong     bối cảnh giới hạn hạ lưu   nào đó.
Trong bản đồ    bối cảnh,     ngôn ngữ được xuất bản kết hợp dịch vụ máy chủ mở    được ký hiệu là     OHS|PL.
 
 
     

% \section{Áp dụng về các mối quan hệ bối cảnh giới hạn}

%%%%%%%%%%%%%%%%%%%%%%%%%%%%%%%%%%

%%%%%%%%%%%%%%%%%%%%%%%%%%%%%%%%%%

%%%%%%%%%%%%%%%%%%%%%%%%%%%%%%%%%%

%%%%%%%%%%%%%%%%%%%%%%%%%%%%%%%%%%

%%%%%%%%%%%%%%%%%%%%%%%%%%%%%%%%%%

%%%%%%%%%%%%%%%%%%%%%%%%%%%%%%%%%%

%%%%%%%%%%%%%%%%%%%%%%%%%%%%%%%%%%

%%%%%%%%%%%%%%%%%%%%%%%%%%%%%%%%%%

% \chapter{Các mẫu kỹ thuật}

% Các mẫu     kỹ thuật được sử dụng để lập mô hình và hiện thực hóa  các thành phần riêng lẻ của hệ thống microservices.  Các mẫu     kỹ thuật  tập trung    mô hình hóa miền và triển khai logic nghiệp vụ trong lập trình.




 

Các yếu tố  các mẫu     kỹ thuật  bao gồm:





\begin{itemize}
\item Muc1  
\item Muc1  
\item Muc1  
\item Muc1  
\item Muc1  
\item Muc1  
\item Muc1  
\item Muc1  
\item Muc1  
\item Muc2  
\end{itemize}






  

%!<! - - $ Vẽ lại sau: - - >

%!<! - - $ Vẽ lại sau: - - >

%!<! - - $ Vẽ lại sau: - - >

%!<! - - $ Vẽ lại sau: - - >

%!<! - - $ Vẽ lại sau: - - >

%!<! - - $ Vẽ lại sau: - - >

%!<! - - $ Vẽ lại sau: - - >

%!<! - - $ Vẽ lại sau: - - >

%!<! - - $ Vẽ lại sau: - - >

%!<! - - $ Vẽ lại sau: - - >

%!<! - - $ Vẽ lại sau: - - >

%!<! - - $ Vẽ lại sau: - - >

%!<! - - $ Vẽ lại sau: - - >

% \section{Các đối tượng miền (Domain Object)}

% Đối tượng miền (Domain Object)    được sử dụng để mô hình hóa    mô hình miền.  Đối tượng miền    được sử dụng để   triển khai   các quy tắc, các ràng buộc và các mối quan hệ   của nghiệp vụ.
Đối tượng miền   bao gồm:





\begin{itemize}
\item Đối tượng thực thể (Entities Objects)  
\item Đối tượng giá trị (Value Objects) 
\item  Miền dịch vụ (Service Domain) 

\end{itemize}

 
 
 

 

% \subsection{Đối tượng thực thể (Entities Objects)}

% % \subsection{Đối tượng thực thể (Entities Objects)}
%%%%%%%%%%%%%%%%%%%%%%%%%%%%%%%%%%
Định nghĩa:
% <!-- Một thực thể đại diện cho một đối tượng kinh doanh có thể nhận dạng duy nhất, bao gồm các thuộc tính và hành vi miền được xác định rõ ràng. -->
\begin{example}
    
\end{example}
% mối quan hệ giữa logic nghiệp vụ và các đối tượng thực thể.


% <!-- Một thực thể được xác định duy nhất trong một bối cảnh giới hạn. -->



% Các thực thể này và danh tính của chúng chỉ có ý nghĩa trong bối cảnh giới hạn tương ứng của chúng. Một thực thể có một tập hợp các thuộc tính được xác định bởi ngôn ngữ chung cho ngữ cảnh giới hạn .
 

% Một thực thể có một hành vi, nghĩa là nó đóng gói logic nghiệp vụ. Và logic kinh doanh này được thể hiện qua cách thức hoạt động.
 

% Khi các hoạt động này được thực hiện đối với thực thể, nó sẽ dẫn đến sự thay đổi trạng thái của thực thể.  

Khi một số thao tác này được thực thi, chúng sẽ thay đổi trạng thái của tài khoản. Ví dụ: số dư có thể tăng hoặc giảm do thực hiện một số thao tác này.

13

00: 02: 23, 100--> 00: 02: 29, 160

Hãy xác định logic kinh doanh là gì, không phải kinh doanh. Logic đôi khi được gọi là logic miền.

14

00: 02: 29, 160--> 00: 02: 38, 680

Logic nghiệp vụ có thể bao gồm các quy tắc nghiệp vụ. Ví dụ: việc rút tiền sẽ không thành công nếu số dư nhỏ hơn số tiền rút.

15

00: 02: 38, 850--> 00: 02: 55, 860

Nó có thể là sự xác nhận. Ví dụ: số tiền rút không được nhỏ hơn hoặc bằng 0. Nó có thể là các phép tính, ví dụ, tính chéo thành phần cho tài khoản séc và nó có thể là các hoạt động có thể thay đổi trạng thái của thực thể.

16

00: 02: 55, 860--> 00: 03: 05, 880

Ví dụ: logic giao dịch rút tiền có thể được kết hợp để thực hiện tất cả những điều này được gọi là logic nghiệp vụ nói chung.

17

00: 03: 06, 330--> 00: 03: 15, 600

Hãy xem một ví dụ về logic nghiệp vụ hoặc hành vi. Trong trường hợp tài khoản séc, có thể có hoạt động rút tiền từ tài khoản.

18

00: 03: 15, 900--> 00: 03: 28, 320

Hoạt động rút tiền này có thể lấy số tiền rút làm đối số. Việc kiểm tra đầu tiên sẽ được thực hiện khi thực hiện thao tác rút tiền là kiểm tra số dư khả dụng.

19

00: 03: 28, 320--> 00: 03: 39, 840

Nếu số dư khả dụng nhỏ hơn số tiền rút thì giao dịch sẽ bị từ chối. Nếu không, giao dịch sẽ được chấp nhận và số dư sẽ bị giảm đi theo số tiền rút.

20

00: 03: 40, 140--> 00: 03: 49, 020

Đã đến lúc làm một bài kiểm tra nhanh. Hãy nhìn vào thực thể tài khoản kiểm tra này. Chúng ta có nghĩ rằng thực thể này bộc lộ một số logic kinh doanh không?

21

00: 03: 50, 700--> 00: 03: 58, 350

Câu trả lời là không, không phải vậy, và lý do tôi nói vậy là vì những thao tác này chỉ là getters và setters.

22

00: 03: 58, 350--> 00: 04: 08, 820

Tương tự, thao tác này là sở hữu đối tượng vào CSDL . Vì vậy, nhìn từ bề ngoài, có vẻ như thực thể này không bao gồm bất kỳ hoạt động kinh doanh nào.

23

00: 04: 08, 820--> 00: 04: 21, 180

Các thực thể logic chỉ có ý nghĩa trong bối cảnh ranh giới mà chúng được xác định. Người ta thường thấy các tên thực thể giống nhau xuất hiện trên nhiều ngữ cảnh được liên kết.

24

00: 04: 21, 630--> 00: 04: 29, 200

Nhưng chúng ta phải nhớ rằng định nghĩa của thực thể trong bối cảnh giới hạn này không được đảm bảo giữ nguyên.

25

00: 04: 29, 310--> 00: 04: 39, 300

Ví dụ: thực thể khách hàng trong tài khoản bán lẻ có thể trông không giống với thực thể khách hàng trong bối cảnh giới hạn thẻ tín dụng.

26

00: 04: 39, 660--> 00: 04: 53, 780

Hãy nhớ rằng các thực thể được xác định duy nhất trong một ngữ cảnh giới hạn, nhưng đôi khi có thể xảy ra trường hợp cùng một thuộc tính được sử dụng để xác định duy nhất thực thể trong các liên hệ công việc.

27

00: 04: 53, 790--> 00: 05: 05, 620

Nhưng đó hoàn toàn là sự trùng hợp ngẫu nhiên. Vì vậy, trong ví dụ này, số An sinh xã hội của khách hàng đang được sử dụng để nhận dạng duy nhất khách hàng trong cả hai địa chỉ liên hệ công việc này.

28

00: 05: 06, 180--> 00: 05: 26, 420

Hãy nhớ rằng, đó thực sự là sự trùng hợp ngẫu nhiên. Việc xác định các thực thể này của nhóm sẽ hoạt động độc lập với nhau và xác định các thuộc tính cũng như hoạt động cho các thực thể dựa trên yêu cầu trong từng bối cảnh nghiệp vụ, các thực thể được lưu trữ lâu dài.

29

00: 05: 26, 700--> 00: 05: 40, 000

Dữ liệu được lưu trữ dài hạn thể hiện trạng thái hiện tại của thực thể. Điều này phổ biến đối với RDBMS và không có CSDL đơn lẻ nào được sử dụng để lưu trữ liên tục các thực thể.

30

00: 05: 40, 620--> 00: 05: 56, 450

Trong trường hợp RDBMS, một bảng biểu thị một tập hợp các thực thể. Các quy tắc trong bảng biểu thị các thực thể được xác định duy nhất bằng cột khóa chính.

31

00: 05: 56, 700--> 00: 06: 06, 740

Các cột còn lại có các giá trị cho thuộc tính của từng thực thể. Đã đến lúc xem lại nhanh những điểm chính của bài giảng này.

32

00: 06: 06, 840--> 00: 06: 14, 640

Điều đầu tiên là các thực thể là các đối tượng kinh doanh chỉ có ý nghĩa trong một bối cảnh giới hạn.

33

00: 06: 14, 640--> 00: 06: 26, 760

Nơi chúng được xác định là các thực thể được xác định duy nhất trong bối cảnh giới hạn . Tiếp theo là định nghĩa của thực thể bao gồm thuộc tính và hành vi.

34

00: 06: 27, 060--> 00: 06: 35, 700

Hành vi này triển khai logic nghiệp vụ có thể thay đổi trạng thái của thực thể. Các thực thể được lưu trữ lâu dài.




%%%%%%%%%%%%%%%%%%%%%%%%%%%%%%%%%%

Đối tượng thực thể (Entities Objects)  là đối tượng miền có có định danh riêng  duy nhất.  Định danh này được giữ nguyên xuyên suốt trạng thái hoạt động của hệ thống phần mềm.


Các thực thể là những đối tượng rất quan trọng của mô hình miền. Việc xác định xem một đối tượng có phải là thực thể hay không rất quan trọng.

Trong trường hợp CSDL quan hệ, một bảng biểu thị một tập hợp các thực thể. Các quy tắc trong bảng biểu thị các thực thể được xác định duy nhất bằng cột khóa chính.

Hành vi này triển khai logic nghiệp vụ có thể thay đổi trạng thái của thực thể. Các thực thể được lưu trữ lâu dài.

% %! Entity : https:// thiết kế hướng miền - practitioners.com/entity

% %! Entity : https:// thiết kế hướng miền - practitioners.com/entity

% %! Entity : https:// thiết kế hướng miền - practitioners.com/entity

% %! Entity : https:// thiết kế hướng miền - practitioners.com/entity


thực thể

Trong Thiết kế hướng miền (thiết kế hướng miền), thực thể là khái niệm cốt lõi đại diện cho một đối tượng miền có nhận dạng duy nhất. Thực thể là một đối tượng được phân biệt với các đối tượng khác dựa trên nhận dạng duy nhất của nó, thay vì thuộc tính hoặc giá trị của nó.

Các thực thể thường là các đối tượng quan trọng nhất trong mô hình miền và chúng thường có logic và hành vi nghiệp vụ phức tạp được liên kết với chúng. Họ cũng có thể có mối quan hệ với các thực thể, đối tượng giá trị hoặc dịch vụ miền khác.

Một thực thể có các đặc điểm sau:

Danh tính: Một thực thể có một danh tính duy nhất để phân biệt nó với các thực thể khác trong mô hình miền. Danh tính thường được biểu thị bằng ID hoặc khóa, chẳng hạn như ID khách hàng hoặc SKU sản phẩm.

Khả năng thay đổi: Các thuộc tính của thực thể có thể thay đổi theo thời gian trong khi vẫn duy trì được danh tính của nó. Ví dụ: tên hoặc địa chỉ của khách hàng có thể thay đổi nhưng ID khách hàng vẫn giữ nguyên.

Hành vi: Một thực thể có hành vi liên quan đến nó, thường là các quy tắc và logic nghiệp vụ phức tạp. Hành vi này thường được gói gọn trong chính thực thể đó.

Mối quan hệ: Một thực thể có thể có mối quan hệ với các thực thể, đối tượng giá trị hoặc dịch vụ miền khác. Ví dụ: khách hàng có thể có lịch sử đặt hàng hoặc giỏ hàng.

Các thực thể là một phần thiết yếu của mô hình miền và phải được thiết kế để thể hiện chính xác miền và các quy tắc kinh doanh của nó. Bằng cách lập mô hình chính xác các thực thể, nhà phát triển có thể tạo ra giải pháp phần mềm linh hoạt và dễ bảo trì hơn, đáp ứng nhu cầu của miền.

Một ví dụ

Hãy xem xét một nền tảng thương mại điện tử nơi khách hàng có thể đặt hàng sản phẩm. Trong mô hình miền này, Đơn hàng là một thực thể. Mỗi đơn hàng có một danh tính duy nhất và bất biến, chẳng hạn như số đơn hàng, giúp phân biệt nó với các đơn hàng khác trong hệ thống.

Thực thể Đơn hàng có thể có một số thuộc tính, chẳng hạn như thông tin khách hàng, chi tiết thanh toán và thông tin giao hàng. Nó cũng có thể có mối quan hệ với các thực thể khác, chẳng hạn như thực thể Sản phẩm và Khách hàng.

Hành vi của thực thể Đơn hàng bao gồm tạo và cập nhật đơn hàng, quản lý xử lý thanh toán và theo dõi trạng thái đơn hàng.

Dưới đây là ví dụ về giao diện của thực thể Đơn hàng trong mã:

public class Order {

private OrderId orderId;

private Customer customer;

private List<Product> products;

private Date orderDate;

private PaymentDetails paymentDetails;

private ShippingDetails shippingDetails;

public Order(OrderId orderId, Customer customer) {

this.orderId = orderId;

this.customer = customer;

this.products = Lists.newArrayList();

this.orderDate = LocalDate.now();

}

public void addProduct(Product product) {

products.add(product);

}

public void removeProduct(Product product) {

products.remove(product);

}

public void processPayment() {

// Process payment logic here...

}

public void shipOrder() {

// Shipping logic here...

}

// Other behavior methods here...

}

Trong ví dụ này, thực thể Đơn hàng có một ID duy nhất (orderId) xác định nó trong hệ thống, cùng với các thuộc tính khác như customer, < /span>, gói gọn logic kinh doanh được liên kết với thực thể đơn hàng., và,, . Thực thể cũng có các phương thức hành vi, chẳng hạn như và, products, orderDatepaymentDetailsshippingDetailsaddProductremoveProductprocessPaymentshipOrder

% %! Entity : https:// thiết kế hướng miền - practitioners.com/entity

% %! Entity : https:// thiết kế hướng miền - practitioners.com/entity

% %! Entity : https:// thiết kế hướng miền - practitioners.com/entity


% %! Entity Identity : https:// thiết kế hướng miền - practitioners.com/entity - identity

% %! Entity Identity : https:// thiết kế hướng miền - practitioners.com/entity - identity

% %! Entity Identity : https:// thiết kế hướng miền - practitioners.com/entity - identity

% %! Entity Identity : https:// thiết kế hướng miền - practitioners.com/entity - identity

Trang chủTrang chủBảng chú giảiNhận dạng thực thể

Nhận dạng thực thể

Danh tính của một thực thể phải là duy nhất và bất biến, nghĩa là nó không được thay đổi trong suốt vòng đời của thực thể đó. Việc thay đổi danh tính của một thực thể có thể gây ra hậu quả nghiêm trọng, chẳng hạn như gây ra sự không nhất quán về dữ liệu hoặc phá vỡ mối quan hệ với các thực thể hoặc đối tượng giá trị khác. Ví dụ: nếu ID của khách hàng bị thay đổi, điều đó có thể dẫn đến nhầm lẫn khi theo dõi lịch sử mua hàng của họ hoặc các tương tác khác với hệ thống.

Điều quan trọng cần lưu ý là các thuộc tính của thực thể, chẳng hạn như tên hoặc địa chỉ, có thể thay đổi mà không ảnh hưởng đến danh tính của thực thể đó. Những thay đổi này phải được quản lý thông qua việc đóng gói thích hợp hành vi và logic kinh doanh của thực thể.

Tóm lại, danh tính của một thực thể phải là duy nhất và bất biến, đồng thời những thay đổi đối với các thuộc tính của thực thể sẽ không ảnh hưởng đến danh tính của thực thể đó. Bằng cách tuân thủ nguyên tắc này, các nhà phát triển có thể tạo ra một mô hình miền nhất quán và dễ bảo trì hơn, thể hiện chính xác miền kinh doanh.

Làm thế nào để chọn một danh tính thực thể tốt

Chọn danh tính phù hợp cho một thực thể là một phần quan trọng trong việc thiết kế mô hình miền trong Thiết kế hướng miền (thiết kế hướng miền). Dưới đây là một số phương pháp hay để chọn danh tính của một thực thể:

Chọn một danh tính duy nhất: Danh tính của một thực thể phải là duy nhất trong mô hình miền và nó không được thay đổi trong suốt vòng đời của thực thể. Danh tính của một thực thể phải được xác định theo yêu cầu kinh doanh, chẳng hạn như mã định danh duy nhất như số sê - ri, ID khách hàng hoặc số an sinh xã hội.

Chọn danh tính ổn định: Danh tính của thực thể phải ổn định, nghĩa là danh tính không được thay đổi theo thời gian. Danh tính ổn định cho phép tính nhất quán và độ chính xác trong mô hình miền và nó có thể ngăn ngừa lỗi trong hệ thống. Ví dụ: nếu ID của khách hàng thay đổi, việc theo dõi lịch sử mua hàng của họ hoặc các tương tác khác với hệ thống có thể gây nhầm lẫn.

Chọn danh tính dễ nhận dạng: Danh tính của thực thể phải dễ nhận dạng, tốt nhất là bởi người đọc. Ví dụ: việc sử dụng UUID hoặc GUID có thể không dễ nhận biết như tên hoặc số ID của khách hàng.

Chọn một danh tính có thể được sử dụng nhất quán trong toàn bộ mô hình miền: Danh tính của một thực thể phải nhất quán trong toàn bộ mô hình miền và nó phải được sử dụng nhất quán trong mọi ngữ cảnh mà thực thể đó được tham chiếu.

Xem xét khả năng mở rộng và hiệu suất: Việc chọn một danh tính có thể mở rộng quy mô và hoạt động tốt cũng rất quan trọng, đặc biệt đối với các hệ thống có khối lượng dữ liệu lớn hoặc thông lượng cao.

Bằng cách làm theo những thực tiễn này, nhà phát triển có thể chọn danh tính phù hợp cho các thực thể thể hiện chính xác mô hình miền và cung cấp giải pháp phần mềm linh hoạt và có thể bảo trì.

% %! Entity Identity : https:// thiết kế hướng miền - practitioners.com/entity - identity

% %! Entity Identity : https:// thiết kế hướng miền - practitioners.com/entity - identity

% %! Entity Identity : https:// thiết kế hướng miền - practitioners.com/entity - identity

% %! Entity Identity : https:// thiết kế hướng miền - practitioners.com/entity - identity



% \subsection{Đối tượng giá trị (Value Objects)}

% \input{contents/doi_tuong_gia_tri_value_objects}

% \subsection{Miền dịch vụ (Service)}

% \input{contents/mien_dich_vu_srv}

% %! Hướng dẫn 7/4

% %! Hướng dẫn 7/5

% \subsubsection{xxxxxxx}

% 

% Vẽ lại bản đồ tiếng Việt
% Vẽ lại bản đồ tiếng Việt
% Vẽ lại bản đồ tiếng Việt
% Vẽ lại bản đồ tiếng Việt
% Vẽ lại bản đồ tiếng Việt
% Vẽ lại bản đồ tiếng Việt
% Vẽ lại bản đồ tiếng Việt
% Vẽ lại bản đồ tiếng Việt
% Từ bản đồ lấy vi dụ cho các mô hình
% Từ bản đồ lấy vi dụ cho các mô hình
% Từ bản đồ lấy vi dụ cho các mô hình
% Từ bản đồ lấy vi dụ cho các mô hình
% Từ bản đồ lấy vi dụ cho các mô hình
% Từ bản đồ lấy vi dụ cho các mô hình
% Từ bản đồ lấy vi dụ cho các mô hình
% Từ bản đồ lấy vi dụ cho các mô hình
% Từ bản đồ lấy vi dụ cho các mô hình
% Từ bản đồ lấy vi dụ cho các mô hình
\begin{example} Trong miền vấn đề ngân hàng,     thẻ tín dụng và khoản vay mua nhà không có mối quan hệ. 
    
    \begin{figure}[H]
        
        \centering
        
        \includegraphics[scale = 0.5]{pictures/mo_hinh_rieng_biet_separate_ways/main.drawio.png}
        
        \caption{Ví dụ  mô hình riêng biệt (Separate Ways)  }
        
    \end{figure}
\end{example} 

% %! $VD: hình giao như 2 tập hợp - - >

%%%%%%%%%%%%%%%%%%%%%%%%%%%%%%%%%%

\end{document} % Kết thúc

% kết luận, tài liệu tham khảo

%%%%%%%%%%%%%%%%%%%%%%%%%%%%%%%%%%

% % %! Aggregates/ /

% % Tổng hợp là đối tượng kinh doanh trung tâm trong Bối cảnh giới hạn của chúng ta và xác định phạm vi nhất quán trong bối cảnh giới hạn đó.

% % Tổng hợp = Mã định danh chính của Bối cảnh giới hạn của chúng ta

% \subsubsection{xxxxxxx}

% % 

% Vẽ lại bản đồ tiếng Việt
% Vẽ lại bản đồ tiếng Việt
% Vẽ lại bản đồ tiếng Việt
% Vẽ lại bản đồ tiếng Việt
% Vẽ lại bản đồ tiếng Việt
% Vẽ lại bản đồ tiếng Việt
% Vẽ lại bản đồ tiếng Việt
% Vẽ lại bản đồ tiếng Việt
% Từ bản đồ lấy vi dụ cho các mô hình
% Từ bản đồ lấy vi dụ cho các mô hình
% Từ bản đồ lấy vi dụ cho các mô hình
% Từ bản đồ lấy vi dụ cho các mô hình
% Từ bản đồ lấy vi dụ cho các mô hình
% Từ bản đồ lấy vi dụ cho các mô hình
% Từ bản đồ lấy vi dụ cho các mô hình
% Từ bản đồ lấy vi dụ cho các mô hình
% Từ bản đồ lấy vi dụ cho các mô hình
% Từ bản đồ lấy vi dụ cho các mô hình
\begin{example} Trong miền vấn đề ngân hàng,     thẻ tín dụng và khoản vay mua nhà không có mối quan hệ. 
    
    \begin{figure}[H]
        
        \centering
        
        \includegraphics[scale = 0.5]{pictures/mo_hinh_rieng_biet_separate_ways/main.drawio.png}
        
        \caption{Ví dụ  mô hình riêng biệt (Separate Ways)  }
        
    \end{figure}
\end{example} 

% %! $VD: hình giao như 2 tập hợp - - >

% \end{document} % kết thúc

% Yêu cầu nghiệp vụ của từng sub

% %

% Sơ đồ if else Đ S

% %

% sub trước model

% %

%%%%%%%%%%%%%%%%%%%%%%%%%%%%%%%%%%%%%

\end{document}

\section{xxxxxxx}

\subsection{xxxxxxx}

\subsubsection{xxxxxxx}



% Vẽ lại bản đồ tiếng Việt
% Vẽ lại bản đồ tiếng Việt
% Vẽ lại bản đồ tiếng Việt
% Vẽ lại bản đồ tiếng Việt
% Vẽ lại bản đồ tiếng Việt
% Vẽ lại bản đồ tiếng Việt
% Vẽ lại bản đồ tiếng Việt
% Vẽ lại bản đồ tiếng Việt
% Từ bản đồ lấy vi dụ cho các mô hình
% Từ bản đồ lấy vi dụ cho các mô hình
% Từ bản đồ lấy vi dụ cho các mô hình
% Từ bản đồ lấy vi dụ cho các mô hình
% Từ bản đồ lấy vi dụ cho các mô hình
% Từ bản đồ lấy vi dụ cho các mô hình
% Từ bản đồ lấy vi dụ cho các mô hình
% Từ bản đồ lấy vi dụ cho các mô hình
% Từ bản đồ lấy vi dụ cho các mô hình
% Từ bản đồ lấy vi dụ cho các mô hình
\begin{example} Trong miền vấn đề ngân hàng,     thẻ tín dụng và khoản vay mua nhà không có mối quan hệ. 
    
    \begin{figure}[H]
        
        \centering
        
        \includegraphics[scale = 0.5]{pictures/mo_hinh_rieng_biet_separate_ways/main.drawio.png}
        
        \caption{Ví dụ  mô hình riêng biệt (Separate Ways)  }
        
    \end{figure}
\end{example} 

% %! $VD: hình giao như 2 tập hợp - - >

% phải có CQRS (Phân chia trách nhiệm truy vấn lệnh)

CQRS là một mẫu kiến trúc riêng biệt có thể được sử dụng kết hợp với thiết kế hướng miền để đạt được những lợi ích nhất định, chẳng hạn như cải thiện hiệu suất và khả năng mở rộng. Tuy nhiên, nó không phải là một yêu cầu để triển khai thiết kế hướng miền.

% phải có event

Ngôn ngữ chung (Ubiquitous Language)

%%%%%%%%%%%%%%%%%%%%%%%%%%%%%%%%%%%%%

\end{document} % kết thúc

Cách tiếp cận này nhấn mạnh tính mô - đun, tính linh hoạt và khả năng phục hồi, cho phép các nhóm làm việc đồng thời trên các phần khác nhau của hệ thống và cho phép phát hành nhanh hơn và thường xuyên hơn. Các vi dịch vụ thường dựa vào các giao thức truyền thông nhẹ, chẳng hạn như REST và thường được triển khai bằng các công nghệ chứa trong bộ chứa như Docker và Kubernetes.

\subsubsection{DevOps Ứng dụng, áp dụng, liên quan,....}

\subsubsection{CI/CD}

\subsubsection{Docker}

\subsubsection{Kubernetes}

dícovery

api gateway

%@ Tất cả phải dùng ulli